\documentclass{article}
\usepackage{fancyhdr}
\usepackage[catalan]{babel}
\usepackage[T1]{fontenc}
\usepackage[utf8]{inputenc}
\usepackage{amssymb}
\usepackage{amsmath}
 
\pagestyle{fancy}
\fancyhf{}


\lhead{Tema 9. Funcions.\\ Exercicis. Operacions amb funcions. Composició de funcions. Funció inversa.}
\lfoot{Institut de Vilafant. Matemàtiques 1 batxillerat científic}
\rfoot{\thepage}
 
\begin{document}
	\author{Mireia Dosil}
	\date{text}



\begin{enumerate}
 

\item Si $f(x)=x^3+2x^2-3x+4$ i $g(x)=x^4-3x^3+x^2-5x+1$:

\begin{enumerate}
	\item Troba $(f+g)(x)$ i el seu domini.
	\item Troba les imatges de $(f+g)$ dels nombres $-2,-1,-0$ i $1$.
	\item Calcula $g \circ f$
\end{enumerate}

\item Si $f(x)=\frac{1}{x-2}$ i $g(x)=\frac{x^2+1}{x^2+4x+3}$:

\begin{enumerate}
	\item Troba $(f+g)(x)$ i el seu domini.
	\item Troba $(f\cdot g)(x)$ i el seu domini.
	\item Troba $(\frac{f}{g})(x)$ i el seu domini.
	\item Calcula $g \circ f$
\end{enumerate}


\item Si $f(x)=\sqrt{x^2-9}$ i $g(x)=\frac{\sqrt{x+1}}{x^2-4}$:

\begin{enumerate}
	\item Troba $(f+g)(x)$ i el seu domini.
	\item Calcula $g \circ f$
\end{enumerate}

\item Si $f(x)=x+4$ i $g(x)=\sqrt{x-3}$:

\begin{enumerate}
	\item Troba $(f+g)(x)$ i el seu domini.
	\item Troba $(f\cdot g)(x)$ i el seu domini.
	\item Calcula $g \circ f$
	\item Calcula $f \circ f$	
	
	
\end{enumerate}



\item Trobeu les correspondències inverses de les funcions següents diguent si són també funcions i donant el seu domini.


\begin{enumerate}
	\item $y=f(x)=3x+5$
	\item $y=f(x)=-5x+\sqrt{2}$
	\item $y=f(x)=\sqrt{x^2-4}$
	
	
\end{enumerate}


\end{enumerate}
 
\end{document}