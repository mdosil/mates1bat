\documentclass{article}
\usepackage{fancyhdr}
\usepackage[catalan]{babel}
\usepackage[T1]{fontenc}
\usepackage[utf8]{inputenc}
\usepackage{amssymb}
 
\pagestyle{fancy}
\fancyhf{}

\rhead{Exercicis. Càlcul de dominis}
\lhead{Tema 9. Funcions.}
\lfoot{Institut de Vilafant. Matemàtiques 1 batxillerat científic}
\rfoot{\thepage}
 
\begin{document}
	\author{Mireia Dosil}
	\date{text}

Calcula el domini de les funcions següents:

\begin{enumerate}
 

\item $f(x)=\frac{3x+1}{2x-4}$ Sol: $D=\mathbb{R}-\{2\}$
\item $f(x)=\frac{2x+1}{x^2-5x+6}$ Sol: $D=\mathbb{R}-\{2,3\}$
\item $f(x)=\sqrt{-x^2+64}$ Sol: $D=[-8,8]$
\item $f(x)=\sqrt{x^2+4x+3}$ Sol: $D=(-\infty,-3] \cup [-1, +\infty)$
\item $f(x)=\sqrt[3]{\frac{3x+4}{1-x}}$ Sol: $D=\mathbb{R}-\{1\}$
\item $f(x)=\frac{1}{2x^2+x+3}$ Sol: $D=\mathbb{R}$
\item $f(x)=\frac{3x}{x^4-5x^2+4}$ Sol: $D=\mathbb{R}-\{-2,-1,1,2\}$
\item $f(x)=\frac{3}{\sqrt{x}}+4x^2$ Sol: $D=\mathbb{R}^+$
\item $f(x)=\frac{2x^2+1}{\sqrt{(x-1)(x-2)}}+4x^2$ Sol: $D=(-\infty,1) \cup (2, +\infty)$
\item $f(x)=\sqrt{x^2+1}+\sqrt[6]{2-x}$ Sol: $D=(-\infty,2]$
\item $f(x)=\sqrt{-x^2-4}$ Sol: $D=\nexists$
\item $f(x)=\sqrt{x^2+x+2}$ Sol: $D=\mathbb{R}$
\end{enumerate}
 
\end{document}