\documentclass{article}
\usepackage{fancyhdr}
\usepackage[catalan]{babel}
\usepackage[T1]{fontenc}
\usepackage[utf8]{inputenc}
 
\pagestyle{fancy}
\fancyhf{}

\rhead{Exercicis}
\lhead{Tema 8. Successions.}
\lfoot{Institut de Vilafant. Matemàtiques 1 batxillerat científic}
\rfoot{\thepage}
 
\begin{document}
	\author{Mireia Dosil}
	\date{text}


\begin{enumerate}
 
\item Demostra que la successió $(\frac{n}{n+1})$ és creixent. És estrictament creixent?
\item Demostra que la successió $(\frac{n+3}{n})$ és decreixent. És estrictament decreixent?
\item De les següents successions, diguies quines són creixents o decreixents:

\begin{enumerate}
	\item $(\frac{3}{n+3})$
	\item $(\frac{2n+1}{2n})$
	\item $(\frac{n^2}{n^2+1})$
	\item $(\frac{2n+4}{1-3n})$
\end{enumerate}
 
\item Prova que la sucessió $(n\cdot (-1)^n)$ no és fitada ni superiorment ni inferiorment. 

\tiny(Pista: cal que demostris que no es pot trobar una fita superior/inferior més gran/petita que tots els termes de la sucessió)

\normalsize
\item Demostra que la successió $(\frac{n}{n+1})$ està fitada superiorment i inferiorment.

\item De les següents successions, digues quines són fitades i quines no.
\begin{enumerate}
	\item $(\frac{n-1}{n})$
	\item $((-1)^n-1)$
	\item $(n^2+1)$
	\item $(\frac{1}{n3})$
	\item $(3^{n-1})$
	\item $(-2n+3)$
	
\end{enumerate}



\end{enumerate}
 
\end{document}